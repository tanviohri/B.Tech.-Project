\documentclass[12pt,a4wide]{report}

\usepackage{amsthm,amssymb,mathrsfs,setspace,pstcol}%amsmath, latexsym,footmisc
\usepackage{subcaption}
\usepackage{play}
\usepackage{epsfig}
 \usepackage{amsmath}
%\usepackage[grey,times]{quotchap}
\usepackage[nottoc]{tocbibind}
\renewcommand{\chaptermark}[1]{\markboth{#1}{}}
\renewcommand{\sectionmark}[1]{\markright{\thesection\ #1}}
%

\input xy
\xyoption{all}


\theoremstyle{plain}
\newtheorem{theorem}{Theorem}[section]
\newtheorem{lemma}[theorem]{Lemma}
\newtheorem{corollary}[theorem]{Corollary}
\newtheorem{proposition}[theorem]{Proposition}

\theoremstyle{definition}
\newtheorem{definition}[theorem]{Definition}
\newtheorem{example}[theorem]{Example}
\newtheorem{notation}[theorem]{Notation}

\theoremstyle{remark}
\newtheorem{remark}[theorem]{Remark}

\renewcommand{\baselinestretch}{1.5}



\begin{document}

%\pagenumbering{arabic} \setcounter{page}{1}

% --------------- Title page -----------------------
\begin{titlepage}
\enlargethispage{3cm}

\begin{center}

\vspace*{-2cm}

\textbf{\Large MANAGING SUPPLY IN THE INDIAN CONTEXT:
FLEXIBLE WORKERS, FULL-TIME EMPLOYEES AND FREELANCERS}

\vfill

 A Project Report Submitted \\
  in partial fulfilment of the Requirements  \\
  for the Degree of\\[10pt]

 {\Large \bf BACHELOR OF TECHNOLOGY}\\
 {\bf in}\\
 {\bf Mathematics and Computing}\\ [5pt]

 \vfill

{\large \emph{by}}\\[5pt]
{\large\bf {Manne Hema Priya \& Tanvi Ohri}}\\[5pt]
{\large (Roll No. 170123032 \& 170123051) }\\[5pt]
\vfill
\includegraphics[height=2.5cm]{iitglogo.eps}

\vspace*{0.5cm}

{\em\large to the}\\[10pt]
{\bf\large DEPARTMENT OF MATHEMATICS} \\[5pt]
{\bf\large \mbox{INDIAN INSTITUTE OF TECHNOLOGY GUWAHATI}}\\[5pt]
{\bf\large GUWAHATI - 781039, INDIA}\\[10pt]
{\it\large April 2021}
\end{center}

\end{titlepage}

\clearpage

% --------------- Certificate page -----------------------
\pagenumbering{roman} \setcounter{page}{2}
\begin{center}
{\Large{\bf{CERTIFICATE}}}
\end{center}
%\thispagestyle{empty}


\noindent
This is to certify that the work contained in this report
entitled {\textbf{``Managing Supply in the Indian Context:
Flexible Workers, Full-Time Employees And Freelancers"}} submitted by \textbf{Manne Hema Priya}
(\textbf{Roll No: 170123032}) and \textbf{Tanvi Ohri}
(\textbf{Roll No: 170123051}) to Department of Mathematics, Indian Institute of Technology
Guwahati towards partial requirement of Bachelor of Technology in
Mathematics and Computing has been carried out by them under my
supervision. \\ \\% It is also certified that this report is a survey work based on the references in the bibliography. \\ \\
%OR% \\
It is also certified that, along with literature survey, a few new results are established by the students under the project.\\ \\
Turnitin Similarity:  18\%

\vspace{4cm}

\noindent Guwahati - 781 039 \hfill (Dr. N.Selvaraju)

\noindent April 2021 \hfill Project Supervisor

\clearpage

% --------------- Abstract page -----------------------
\begin{center}
{\Large{\bf{ABSTRACT}}}
\end{center}


The main aim of the project is to adapt the model proposed by Dong and Ibrahim (2020) \cite{dong} for the Indian market.There is a rise in the number of firms that are staffing a blended workforce. The model proposed in Dong and Ibrahim (2020) \cite{dong} considers two types of employees the fixed and flexible. With the increasing trend of startups in India, we feel we need to include another type of employee- the freelancer. The flexible employee is hired for a short interval, say when we typically expect higher traffic or a certain kind of work. The freelancers are hired for a particular task, that is, the profession's or startup's area of expertise. Our goal is to devise an optimal staffing strategy to combat supply-side uncertainty which staffs freelancers, flexible workers and full-time employees. The strategy must take into account both quality and efficiency factors. We first present a comprehensive survey of some of the existing works in this area, namely staffing under parameter uncertainty \cite{bassamboo} and staffing with self-scheduling servers \cite{ibrahim}. Then we move on to combine both the concepts and adapt the model proposed by Dong and Ibrahim in \cite{dong} ,adding a new type of employee to the model- the freelancer.

\clearpage



\tableofcontents
\clearpage
\listoffigures
\listoftables


\newpage

\pagenumbering{arabic}
\setcounter{page}{1}

% =========== Main chapters starts here. Type in separate files and include the filename here. ==
% ============================

\input chapter0.tex

\input chapter1.tex

\input chapter2.tex

\input chapter3.tex

\input chapter4.tex

\input chapter5.tex
\clearpage
% --------------- Conclusion -----------------------
%\begin{center}
%{\Large{\bf{CONCLUSION}}}
%end{center}



\clearpage


\bibliographystyle{plain}
\bibliography{bib}

\end{document}

